% TeX encoding = utf8
% TeX spellcheck = pl_PL 
\documentclass[a4paper, 12pt]{article}
\usepackage[utf8]{inputenc}
\usepackage[polish]{babel}
\usepackage{polski}
\usepackage{float}
\usepackage{graphicx}
\usepackage{listings}
\usepackage{amsfonts}
\usepackage{geometry}
\usepackage{indentfirst}
\usepackage{subfigure}
\usepackage{url}
\usepackage{listings}
\usepackage{color}
\usepackage[usenames,dvipsnames]{xcolor}

%\renewcommand*{\addcontentsline}[3]{\addtocontents{#1}{\protect\contentsline{#2}{#3}{}}}
\newgeometry{tmargin=2.5cm, bmargin=2.5cm, lmargin=3.5cm, rmargin=2.5cm}
%\setcounter{secnumdepth}{2}
\setlength{\fboxsep}{0pt}
\lstset{
	basicstyle=\footnotesize\ttfamily,
	breaklines=true,
	language=Python,
	breakatwhitespace=true,
	frame=leftline,
	numbers=left,
	numberstyle=\tiny,
	commentstyle=\color{Gray}\footnotesize\ttfamily}


\author{Anna Wujek \\ Łukasz Korpal \\ Wiktor Ślęczka}
\title{Roboty monitorujace skażenie środowiska - symulator V-REP}


\begin{document}
	\sloppy
	\maketitle
	\newpage
	\tableofcontents
	\newpage
	\section{Zadanie}
	Celem projektu było stworzenie symulatora mobilnej sieci ad hoc (MANET) do monitorowania skażenia środowiska naturalnego. Węzłami sieci są roboty mobilne, wyposażone w czujniki oraz komunikujące się między sobą. Zadaniem robotów jest lokalizacja chmury skażenia i otoczenie jej robotami, monitorującymi jej położenie i granice, przy założeniu utrzymania spójności sieci.
	
	
	\section{Projekt systemu}
	\subsection{Założenia}
	\begin{itemize}
		\item Roboty potrafią się na bieżąco lokalizować.
		\item System składa się z pewnej liczby robotów mobilnych.
		\item Roboty mobilne działają autonomicznie i potrafią same zorganizować sieć.
		\item Roboty są wyposażone w czujniki pozwalające im wykrywać poziom skażenia oraz przeszkody.
		\item Roboty są wyposażone w urządzenia pozwalające im się między sobą komunikować się między sobą.
		\item Chmura skażenia jest odgórnie określona i niezmienna w trakcie trwania symulacji.
		\item Nieznana jest mapa przestrzeni roboczej, ale znane są jej granice i kształt - prostokąt.
	\end{itemize}
	
	\subsection{Scenariusz działania}	
		
		Algorytm realizowany przez każdego robota:
		\begin{enumerate}
		\item Warunek początkowy: roboty znajdują się w dowolnym punkcie przestrzeni na mapie. Z punktu tego musi istnieć droga prowadząca do obszaru poszukiwań.
		\item Przemieść się do punktu początkowego, zależnego od liczby robotów, wymiarów przestrzeni poszukiwań i zasięgu czujników, zgodnie ze wzorem:		
		\begin{eqnarray}
		max_spaces &=& sensor_range \cdot precision_range \\
		x &=& -length / 2 \\
		y &=& width / 2 + distances * robot_number + distances / 2
		\end{eqnarray}
		\item Wyznacz kolejny punkt docelowy, zgodnie ze wzorem:
		\begin{eqnarray}
		fvvdfv&=& vfdknvdk
		\end{eqnarray}
		\item Udaj się do punktu docelowego, monitorując skażenie oraz wykrywając przeszkody.
		\item Jeśli wykryto przeszkodę, omiń ją algorytmem Bug2. Wyznacz następny punkt docelowy, jest aktualny został minięty.
		\item Jeśli wykryto chmurę skażenia, zatrzymaj się.
		\item Jeśli cały obszar poszukiwań sprawdzony, to koniec algorytmu; jeśli nie, to przejdź do punktu 3.
		\end{enumerate}
		
	
	\subsection{Struktura systemu}
	System prezentowany w projekcie składa się z następujących elementów:
	\begin{itemize}
	\item symulator przestrzeni roboczej	
	\item symulator robotów
	\item symulator chmury skażenia	
	\item komunikacja
	
	zapewnia przekazywanie informacji pomiędzy poszczególnymi symulatorami.
	\end{itemize}
	
	\subsubsection{Symulator przestrzeni roboczej}
	Symulator ten jest odpowiedzialny za wizualizację całej symulacji - przestrzeni roboczej, przeszkód, robotów oraz chmury skażenia. Generuje także fizyczne oddziaływania między elementami, np. kolizje oraz odczyty czujników (oprócz czujników skażenia).
	
	\subsubsection{Symulator chmury skażenia}
	Symulator ten jest odpowiedzialny za generowanie odczytów czujników skażenia. Odczyt jest obliczany na podstawie założonego wcześniej kształtu chmury oraz sposobu, w jaki skażenie rośnie wgłąb chmury.
	
	\subsubsection{Symulator robota}
	W systemie symulator ten będzie zwielokrotniony, aby uzyskać grupę kilku robotów. Odpowiedzialny jest za wyznaczanie sterowania robotów w zależności od odczytów z czujników i wewnętrznego imperatywu, czyli zadania, jakie roboty realizują.
	
	Każdy robot realizuje dokładnie ten sam algorytm
	
	\subsubsection{Komunikacja}
	Aby cały system działał poprawnie, poszczególne elementy muszą móc się ze sobą komunikować. W systemie rozróżniamy następujące kanały komunikacyjne i przesyłane informacje:
	\begin{itemize}
	\item Symulator przestrzeni roboczej $\rightarrow$ Symulator robota:
		\begin{itemize}
		\item odczyty z czujników (oprócz czujnika skażenia)
		\item pozycja robota (położenie + orientacja)	
		\end{itemize}
	\item Symulator robota $\rightarrow$ Symulator przestrzeni roboczej:
		\begin{itemize}
		\item sterowanie poszczególnymi silnikami
		\end{itemize}
	\item Symulator chmury skażenia $\rightarrow$ Symulator robota:
		\begin{itemize}
		\item odczyt czujnika skażenia
		\end{itemize}
	\item Symulator robota $\rightarrow$ Symulator robota:
		\begin{itemize}
		\item ostatni punkt docelowy, do którego udało się dotrzeć
		\item informacja o wykrytej chmurze skażenia
		\end{itemize}
	
	\end{itemize}
	
	\begin{figure}[h!]
	\centering
	\includegraphics[width=0.6\columnwidth]{img/system.pdf}
	\end{figure}
	
	
	\subsection{Algorytmy}
		
		\subsubsection{Algorytm poszukiwania chmury}
		Roboty rozpoczynają pracę od ustawienia się w formacji linii. Następnie dla każdego robota wyznaczany jest następny punkt docelowy, do którego musi dojechać. Punkty docelowe wyznaczane są w taki sposób, aby roboty ponownie ustawiły się w linii. Po dojechaniu do punktu docelowego robot wysyła informację o tym zdarzeniu do innych robotów. Linia punktów docelowych jest jednocześnie miejscem synchronizacji - jeśli jakiś robot jechał wolniej (np. z powodu omijania przeszkody), to reszta na niego czeka. Powoduje to wolniejsze przeszukiwanie, ale zapewnia spójność sieci -- robot nigdy nie jest oddalony od swoich sąsiadów bardziej, niż odległość do następnego punktu docelowego. Odległość ta dobrana jest w taki sposób, aby spójność sieci została zachowana.
		
		Roboty w ten sposób poruszają się do przodu do momentu dotarcia do przeszkody, toksycznej chmury lub granicy obszaru poszukiwań. W przypadku powyższych wydarzeń, uruchamiane są odpowiednie algorytmy.
		
		Gdy roboty dotrą do końca obszaru poszukiwań, zatrzymują się, przesuwają w bok i kontynuują poszukiwania w drugą stronę, granicząc z poprzednim obszarem poszukiwań -- w ten sposób szerokimi pasami przeszukują cały obszar mapy.
		
		\subsubsection{Algorytm bug 2}
		Algorytm ten jest stosowany w przypadku napotkania przeszkody przez robota. Polega on na podążaniu wzdłuż ściany przeszkody do momentu osiągnięcia punktu leżącego za napotkaną ścianą, na przedłużeniu początkowej ścieżki ruchu. Jego działanie zostało przedstawione na rysunku \ref{bug2_img}.
		\begin{figure}[h!]
			\centering
			\includegraphics*[width=0.7\columnwidth]{img/40-0.png}
			\caption{Działanie algorytmu bug 2}
			\label{bug2_img}
		\end{figure}

		\subsubsection{Algorytm otaczania chmury}
		W momencie napotkania chmury przez któregokolwiek robota, zatrzymuje się on (po osiągnięciu natężenia toksyn określonego arbitralnie). Pozostałe roboty jadą dalej, po łuku -- tak, aby nie przerwać połączenia z siecią. Gdy napotkają określone stężenie chmury, zatrzymują się. Gdy wszystkie roboty zatrzymały się, zwracają się w jedną stronę i zaczynają okrążać chmurę, utrzymując się na granicy stężenia oraz zachowując spójność sieci. Jego działanie przedstawiono na rysunku \ref{otaczanie}.
		\begin{figure}[h!]
			\includegraphics*[width=0.5\columnwidth]{img/chmura/1.png}
			\includegraphics*[width=0.5\columnwidth]{img/chmura/2.png}
			\includegraphics*[width=0.5\columnwidth]{img/chmura/3.png}
			\includegraphics*[width=0.5\columnwidth]{img/chmura/4.png}
			\caption{Działanie algorytmu otaczania chmury. Robot 2 napotyka chmurę, pozostałe roboty zbliżają się do chmury aż do wartości granicznej, następnie cała sieć zaczyna krążyć wokół chmury, monitorując jej kształt i położenie.}
			\label{otaczanie}
		\end{figure}
	
	
	
	\section{Realizacja}
	
	\subsection{Komunikacja}
	Komunikacja przebiega na zasadzie ogłaszania aktualnych informacji na swój temat przez roboty (oraz chmurę). Informacje przesyłane są za pomocą struktury JSON do gniazd, przypisywanych każdemu modułowi osobno.
	Wiadomości opatrzone są nagłówkiem, pozwalającym sklasyfikować komunikat. Dane tworzone są w postaci usystematyzowanych słowników, których struktura została odgórnie ustalona.
	Komunikaty są wysyłane i odbierane asynchronicznie.
	\subsubsection{Roboty}
	Poniżej przedstawiono przykład 
	\begin{verbatim}
	    def broadcast_info(self):
	        #print("Sending info!" + str(self.number))
	        message = str(self.robot._name)+str(self.substage)
	        message = dict()
	        message["robot"] = self.robot.name
	        message["stage"] = self.stage
	        message["substage"] = self.substage
	        message["type"] = "progress"
	        self.robot.commutron.broadcast(json.dumps(message))
	\end{verbatim}
	\subsubsection{Chmura}
	
	
	\subsection{Roboty}
	Wykorzystywane w realizacji zadania roboty Pionieer p3dx wyposażone w ultradźwiękowe czujniki odległości - będą one zapewnione przez symulator V-Rep. Posiadają one napęd różnicowy, pozwalający na dużą swobodę poruszania się po symulowanej przestrzeni. Są przystosowane do rozbudowy o dodatkowe urządzenia -- dzięki czemu mogły by być łatwo dostosowane do realizacji zadania lokalizacji chmury.
	Dodatkowo, czujniki stężenia toksycznych substancji będą realizowane w skryptach Pythona, ułatwiając jej symulację. Informacje o jej położeniu i kształcie będą przekazywane do Vrepa, gdzie nastąpi ich wizualizacja.
	
\end{document}


